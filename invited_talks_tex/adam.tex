%File SyntaxFest2019.tex
\documentclass[11pt]{article}

\usepackage{SyntaxFest2019}
\usepackage{url}
\usepackage{latexsym}
\usepackage{mathptmx}
\usepackage{expex}
\usepackage[T1]{fontenc}
\usepackage{pinyin}

\usepackage{CJKutf8}
\newcommand{\Chinese}[1]{\begin{CJK*}{UTF8}{gbsn}#1\end{CJK*}}


\usepackage{enumitem}
\setlist[itemize]{noitemsep, label={\large\textbullet}}
% You can expand the titlebox if you need extra space
% to show all the authors. Please do not make the titlebox
% smaller than 5cm (the original size); we will check this
% in the camera-ready version and ask you to change it back.


%@@marie: can't succeed in adding a page foot

%\usepackage[footsepline,plainfootsepline]{scrlayer-scrpage}
%\setkomafont{pagefoot}{\scriptsize\itshape}
%\cfoot*{Blablabla}

%% \usepackage{fancyhdr}
%% \pagestyle{fancy}
%% \renewcommand\headrulewidth{1pt}
%% \fancyfoot[C]{SyntaxFest - August 26-30 2019 - Paris}

    %% \usepackage{fancyhdr}
    %% \fancyhead{}% efface le contenu de l'en-tete
    %% \fancyfoot{}% efface le contenu du pied de page
    %% \fancyhead[RO,LE]{\textbf{2009-2010}}
    %% \fancyhead[LO,RE]{\textbf{\nouppercase{\leftmark}}}
    %% \rfoot{\textit{claire Latex}}% pied de page en bas à droite sur la premiere page seulement
    %% \cfoot{\thepage}
    %% \pagestyle{fancy}

\title{{\footnotesize SyntaxFest 2019 - 26-30 August - Paris}\\
\vspace{10mm}
 Invited Talk\\
  {\small Friday 30th August 2019}\\
  Arguments and adjuncts}

\author{Adam Przepiórkowski\\
  University of Warsaw / Polish Academy of Sciences / University of Oxford}
%\\
%  \texttt{email@domain} \\


%\date{}



\begin{document}
\maketitle
\begin{abstract}
Linguists agree that the phrase “two hours” is an argument in “John only lost two hours” but an adjunct in “John only slept two hours”, and similarly for “well” in “John behaved well” (an argument) and “John played well” (an adjunct). While the argument/adjunct distinction is hard-wired in major linguistic theories, Universal Dependencies eschews this dichotomy and replaces it with the core/non-core distinction. The aim of this talk is to add support to the UD approach by critically examininig the argument/adjunct distinction. I will suggest that not much progress has been made during the last 60 years, since Tesnière used three pairwise-incompatible criteria to distinguish arguments from adjuncts. This justifies doubts about the linguistic reality of this purported dichotomy. But – given that this distinction is built into the internal machinery and/or resulting representations of perhaps all popular linguistic theories – what would a linguistic theory not making such an argument–adjunct distinction look like? I will briefly sketch the main components of such an approach, based on ideas from diverse corners of linguistic and lexicographic theory and practice.
\end{abstract}

\vspace{4mm}
\begin{shortbio}
  Adam Przepiórkowski is a full professor at the University of Warsaw (Institute of Philoslphy) and at the Polish Academy of Sciences (Institute of Computer Sciences). As a computational and corpus linguist, he has led NLP projects resulting in the development of various tools and resources for Polish, including the National Corpus of Polish and tools for its manual and automatic annotation, and has worked on topics ranging from deep and shallow syntactic parsing to corpus search engines and valency dictionaries. As a theoretical linguist, he has worked on the syntax and morphosyntax of Polish (within Head-driven Phrase Structure Grammar and within Lexical-Functional Grammar), on dependency representations of various syntactic phenomena (within Universal Dependencies), and on the semantics of negation, coordination and adverbial modifcation (at different periods, within Glue Semantics, Situation Semantics and Truthmaker Semantics). He is currently a visiting scholar at the University of Oxford.
  
\end{shortbio}

\end{document}
