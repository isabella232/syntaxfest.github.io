%File SyntaxFest2019.tex
\documentclass[11pt]{article}

\usepackage{SyntaxFest2019}
\usepackage{url}
\usepackage{latexsym}
\usepackage{mathptmx}
\usepackage{expex}
\usepackage[T1]{fontenc}
\usepackage{pinyin}

\usepackage{CJKutf8}
\newcommand{\Chinese}[1]{\begin{CJK*}{UTF8}{gbsn}#1\end{CJK*}}


\usepackage{enumitem}
\setlist[itemize]{noitemsep, label={\large\textbullet}}
% You can expand the titlebox if you need extra space
% to show all the authors. Please do not make the titlebox
% smaller than 5cm (the original size); we will check this
% in the camera-ready version and ask you to change it back.


%@@marie: can't succeed in adding a page foot

%\usepackage[footsepline,plainfootsepline]{scrlayer-scrpage}
%\setkomafont{pagefoot}{\scriptsize\itshape}
%\cfoot*{Blablabla}

%% \usepackage{fancyhdr}
%% \pagestyle{fancy}
%% \renewcommand\headrulewidth{1pt}
%% \fancyfoot[C]{SyntaxFest - August 26-30 2019 - Paris}

    %% \usepackage{fancyhdr}
    %% \fancyhead{}% efface le contenu de l'en-tete
    %% \fancyfoot{}% efface le contenu du pied de page
    %% \fancyhead[RO,LE]{\textbf{2009-2010}}
    %% \fancyhead[LO,RE]{\textbf{\nouppercase{\leftmark}}}
    %% \rfoot{\textit{claire Latex}}% pied de page en bas à droite sur la premiere page seulement
    %% \cfoot{\thepage}
    %% \pagestyle{fancy}

\title{Invited Talk\\
  {\small Wednesday 28th August 2019}\\
  Transferring NLP models across languages and domains}

\author{Barbara Plank\\
  IT University of Copenhagen}
%\\
%  \texttt{email@domain} \\


%\date{}



\begin{document}
\maketitle
\begin{abstract}
  How can we build Natural Language Processing models for new domains
  and new languages?

  In this talk I will survey some recent advances to
address this ubiquitous challenge, from cross-lingual transfer to
learning models under distant supervision from disparate sources,
multitask-learning and data selection.
\end{abstract}

\vspace{4mm}
\begin{shortbio}
  Barbara Plank is Associate Professor in Natural Language Processing at
IT University of Copenhagen.
She has previously held positions as assistant professor at the
University of Groningen and the University of
Copenhagen, and a postdoc position at the University of Trento. Her
research interests within NLP are broad and include learning under
sample
selection bias (domain adaptation, transfer learning), learning from
beyond the text and multimodal inputs, and in general learning under
limited supervision
for cross-domain and cross-lingual NLP, applied to a range of
applications from author profiling, syntactic language understanding,
information extraction and visual question answering.

Barbara is
member of the advisory board of the Association for Computational
Linguistics and publicity director of the Association for
Computational Linguistics.
\end{shortbio}

\end{document}
